\documentclass[11pt]{article}
\usepackage{mathbbol}
\usepackage{amsmath}
\usepackage{amssymb}
\usepackage{ragged2e}
\usepackage{tikz}
\usepackage{array}
\usepackage{booktabs}
\usepackage[table,xcdraw]{xcolor}
\usetikzlibrary{positioning}
\begin{document}
\noindent Elliot F. Poirier
\hfill
McGill University
\\
Summer 2024
\hfill
MATH470 : Honours Research Project
\\
\begin{center}
\textbf{\LARGE{Using optimization methods in the context of mine planning}}
\end{center}
\hfill
\\


\noindent \textbf{\large{1. Transshipment Problem}} \\

\begin{tikzpicture}[->,>=stealth',shorten >=1pt,auto,node distance=3cm,
    thick,main node/.style={circle,fill=blue!20,draw,font=\sffamily\Large\bfseries}]

\node[main node] (O1) {O1};
\node[main node] (O2) [below of=O1] {O2};
\node[main node] (J1) [right of=O1] {J1};
\node[main node] (J2) [right of=O2] {J2};
\node[main node] (J3) [right of=J1, yshift=-1cm] {J3}; % Adjusted position of J3
\node[main node] (D1) [above right of=J3] {D1}; % Adjusted position of D1
\node[main node] (D2) [below right of=J3] {D2}; % Adjusted position of D2

\node [left of=O1, xshift=2cm] {10}; % Supply at O1
\node [left of=O2, xshift=2cm] {15}; % Supply at O2
\node [right of=D1, xshift=-2cm] {12}; % Demand at D1
\node [right of=D2, xshift=-2cm] {13}; % Demand at D2

\path[every node/.style={font=\sffamily\small}]
(O1) edge node [left] {5} (O2)
(O1) edge node [above] {5} (J1)
(O2) edge node [above] {3} (J2)
(O2) edge node [right] {4} (J1)
(J1) edge node [above] {2} (D1)
(J2) edge node [above] {2} (D2)
(J1) edge node [right] {4} (J3)
(J2) edge node [right] {6} (J3)
(J3) edge node [above] {7} (D1)
(J3) edge node [below] {8} (D2);
\end{tikzpicture}

\noindent We can formulate this problem as a linear program in the following way, 


\begin{aligned}
    \text{min} &\quad \sum_{u,v \in V} t_{u,v}x_{u,v} \\
    \text{s.t.} &\quad \sum_{v \in V} x_{u,v} - \sum_{v \in V} x_{v,u} = a_u \quad \forall u \in V \\ 
    & \quad \sum_{v \in V} x_{v,u} - \sum_{v \in V} x_{u,v} = b_u \quad \forall u \in V \\
    & \quad \sum_{i=1}^m a_i = \sum_{j=1}^n b_j \\
    & \quad x_{u,v} \geq 0 \quad \forall u,v \in V
    \end{aligned}

\noindent consider the following example,\\

\begin{tikzpicture}[->, >=stealth', shorten >=1pt, auto, node distance=4cm, semithick]

    % Styles
    \tikzstyle{node_style} = [circle, draw=black, fill=blue!30, font=\sffamily\Large\bfseries]

    % Nodes
    \node[node_style] (O1) at (-10,8) {O1};
    \node[node_style] (O2) at (-10,4) {O2};
    \node[node_style] (O3) at (-10,0) {O3};
    
    \node[node_style] (J1) at (-4,10) {J1};
    \node[node_style] (J2) at (-4,6) {J2};
    \node[node_style] (J3) at (-4,2) {J3};
    \node[node_style] (J4) at (-4,-2) {J4};
    \node[node_style] (J5) at (-4,-6) {J5};
    
    \node[node_style] (D1) at (2,8) {D1};
    \node[node_style] (D2) at (2,4) {D2};
    \node[node_style] (D3) at (2,0) {D3};
    \node[node_style] (D4) at (2,-4) {D4};
  
    % Origin to Junction connections
    \path (O1) edge node {324} (J1);
    \path (O1) edge node {286} (J2);
    
    \path (O2) edge node {373} (J1);
    \path (O2) edge node [below] {212} (J2);
    \path (O2) edge node {570} (J3);
    \path (O2) edge node {609} (J4);
    
    \path (O3) edge node {658} (J1);
    \path (O3) edge node {405} (J3);
    \path (O3) edge node {419} (J4);
    \path (O3) edge node {158} (J5);
  
    % Junction to Destination connections
    \path (J1) edge node {503} (D1);
    \path (J1) edge node {234} (D2);
    \path (J1) edge node [below] {329} (D3);
    
    \path (J2) edge node {505} (D1);
    \path (J2) edge node {407} (D2);
    \path (J2) edge node {683} (D3);
    
    \path (J3) edge node {398} (D1);
    \path (J3) edge node {253} (D2);
    \path (J3) edge node {171} (D3);
    
    \path (J4) edge node {329} (D2);
    \path (J4) edge node [below] {464} (D3);
    \path (J4) edge node {117} (D4);
    
    \path (J5) edge node [below] {647} (D1);
    \path (J5) edge node {501} (D2);
    \path (J5) edge node {293} (D3);
    \path (J5) edge node {482} (D4);
  
    % Output values (supply)
    \node at (-11,8) {75};
    \node at (-11,4) {125};
    \node at (-11,0) {100};
  
    % Allocation values (demand)
    \node at (3,8) {80};
    \node at (3,4) {65};
    \node at (3,0) {70};
    \node at (3,-4) {85};
  
\end{tikzpicture}

\\
 
\noindent Additional example.\\

\noindent \begin{table}[h!]
    \centering
    \begin{tabular}{|c|c|c|c|}
    \hline

         & WH 1    & WH 2       &  Capacity        \\ \hline
    Plant 1 & 6    & 5          & 600      \\ \hline
    Plant 2 & 4    & 7          & 500      \\ \hline
    Plant 3 & 8    & 5          & 700      \\ \hline
    \end{tabular}
    \end{table}\\
    \begin{table}[h!]
    \begin{tabular}{|c|c|c|c|}
    \hline
    
           & DC 1    & DC 2       & DC 3    \\ \hline
    WH 1   & 6       & 7          & 9       \\ \hline
    WH 2   & 3       & 6          & 12      \\ \hline
    Demand & 400     & 600        & 800     \\ \hline
    \end{tabular}
    \end{table}\\

    \begin{tikzpicture}[->, >=stealth', shorten >=1pt, auto, node distance=4cm, semithick]

        % Styles
        \tikzstyle{node_style} = [circle, draw=black, fill=blue!30, font=\sffamily\Large\bfseries]
    
        % Nodes
        \node[node_style] (O1) at (-10,4) {O1};
        \node[node_style] (O2) at (-10,0) {O2};
        \node[node_style] (O3) at (-10,-4) {O3};
        
        \node[node_style] (J1) at (-4,2) {J1};
        \node[node_style] (J2) at (-4,-2) {J2};
        
        \node[node_style] (D1) at (2,4) {D1};
        \node[node_style] (D2) at (2,0) {D2};
        \node[node_style] (D3) at (2,-4) {D3};
        
        % Labels for supply and demand
        \node at (-11,4) {600};
        \node at (-11,0) {500};
        \node at (-11,-4) {700};
      
        \node at (3,4) {400};
        \node at (3,0) {600};
        \node at (3,-4) {800};
        
        % Origin to Junction connections
        \path (O1) edge node {6} (J1);
        \path (O1) edge node {5} (J2);
        
        \path (O2) edge node {4} (J1);
        \path (O2) edge node {7} (J2);
        
        \path (O3) edge node {8} (J1);
        \path (O3) edge node {5} (J2);
      
        % Junction to Destination connections
        \path (J1) edge node {6} (D1);
        \path (J1) edge node {7} (D2);
        \path (J1) edge node {9} (D3);
        
        \path (J2) edge node {3} (D1);
        \path (J2) edge node {6} (D2);
        \path (J2) edge node {12} (D3);
        
    \end{tikzpicture}\\

  \begin{align}
        \text{min} & \quad 6 x_{o_1,j_1}+5 x_{o_1,j_2}+4 x_{o_2, j_1}+7 x_{o_2,j_2}+8 x_{o_3,j_1}+5 x_{o_3,j_2} \nonumber \\ 
        & \quad +6 x_{j_1,d_1}+7 x_{j_1, d_2}+9 x_{j_1,d_3}+3 x_{j_2,d_1}+6 x_{j_2,d_2}+12 x_{j_2,d_3} \nonumber \\ 
        
        \text{s.t.} & \quad x_{o_1,j_1}+x_{o_1,j_2} \leq 600\\
        & \quad x_{o_2,j_1}+x_{o_2,j_2} \leq 500\\ 
        & \quad x_{o_3,j_1}+x_{o_3,j_2} \leq 700\\
        & \quad x_{o_1, j_1}+x_{o_2, j_1}+x_{o_3,j_1}-x_{j_1,d_1}-x_{j_1,d_2}-x_{j_1,d_3}=0\\
        & \quad x_{o_1, j_2}+x_{o_2, j_2}+x_{o_3,j_2}-x_{j_2,d_1}-x_{j_2,d_2}-x_{j_2,d_3}=0\\
        & \quad x_{j_1,d_1}+x_{j_2,d_1}=400\\
        & \quad x_{j_1,d_2}+x_{j_2,d_2}=600\\
        & \quad x_{j_1,d_3}+x_{j_2,d_3}=800\\
        & \quad x_{u,v} \geq 0 \quad \forall u,v \in V
        \end{align}
    

\noindent \textbf{\large{2. Multiple Vehicle Routing}} \\

\begin{aligned}
    \min \quad & \sum_{u \in V} \sum_{v \in V} w_{u,v} x_{u,v} \\
    \text{s.t} \quad & \sum_{v \in V} x_{v, u}=1 \quad \forall u \in V \backslash\{0\} \\
    & \sum_{u \in V} x_{v, u}=1 \quad \forall v \in V \backslash\{0\} \\
    & \sum_{v \in V \backslash\{0\}} x_{u, 0}= \sum_{u \in V\{0\}} x_{0, v}=k\\
    & \sum_{v \notin S} \sum_{u \in S} x_{v, u} \geq r(S), \quad \forall S \subseteq V \backslash\{0\}, S \neq \emptyset \\
    & x_{v, u} \in\{0,1\} \quad \forall v, u \in V
    \end{aligned}\\

    \noindent The 4th constraint can be rewritten as follows,
    $$
    \sum_{v \in S} \sum_{u \in S} x_{v, u} \leq|S|-r(S)
    $$

\noindent \textbf{\large{3. Multidivisional problem}} \\

\begin{align}
    \text{max} & \quad 8 x_1+5 x_2+6 x_3+9 x_4+7 x_5+9 x_6+6 x_7+5 x_8  + 6 x_9 \nonumber \\ 
    \text{s.t.} & \quad 5 x_1+ 3 x_2+4 x_3+2 x_4+7 x_5+3 x_6+4 x_7+6 x_8 + x_9 \leq 30 \nonumber \\
    &\quad 2 x_1+4 x_2+3 x_3 \leq 5\nonumber\\ 
    & \quad  2x_4+8 x_5+6 x_6 \leq 6 \nonumber\\
    & \quad  3x_7+5 x_8+9 x_9 \leq 32 \nonumber\\
    & \quad x_i \geq 0 \quad \forall i \in \{1, \cdots,9\} \quad \text{and} \quad x_i \in \mathbb{Z} \quad \forall i \in \{1, \cdots,9\} \nonumber
    \end{align} \nonumber

    \noindent We then do the benders decomposition, with initial master problem,\\

\begin{align}
\text{max} & \quad 8 x_1+5 x_2+6 x_3+9 x_4+7 x_5+9 x_6+6 x_7+5 x_8  + 6 x_9 \nonumber \\ 
\text{s.t.} & \quad 5 x_1+ 3 x_2+4 x_3+2 x_4+7 x_5+3 x_6+4 x_7+6 x_8 + x_9 \leq 30 \nonumber \\
\end{align} \nonumber

\noindent and the following subproblems,\\

$$
\begin{aligned}
    \max \quad & \pi_1\left(5 x_1+3 x_2+4 x_3\right)+\pi_2\left(2 x_1+4 x_2+3 x_3\right) \\
    \text { s.t. } \quad  & 2 x_1+4 x_2+3 x_3 \leq 5 \\
    \quad & x_1, x_2, x_3 \geq 0,
 \quad x_1, x_2, x_3 \in \mathbb{Z}
    \end{aligned} \nonumber
$$\\

$$
\begin{aligned}
    \max \quad & \pi_3\left(2 x_4+8 x_5+6 x_6\right) + \pi_4 (2x_4 +7x_5 + 3x_6) \\
    \text { s.t. } \quad  & 2 x_4+8 x_5+6 x_6 \leq 6 \\
    \quad & x_4, x_5, x_6 \geq 0,
 \quad x_4, x_5, x_6 \in \mathbb{Z}
    \end{aligned} \nonumber
    $$\\

$$
\begin{aligned}
    \max \quad & \pi_5\left(3 x_7+5 x_8+9 x_9\right) + \pi_6 (4x_7 +6x_8 + x_9) \\
    \text { s.t. } \quad  & 3 x_7+5 x_8+9 x_9 \leq 32 \\
    \quad & x_7, x_8, x_9 \geq 0,
 \quad x_7, x_8, x_9 \in \mathbb{Z}
    \end{aligned} \nonumber



\noindent After solving the subproblems, we get the following solutions,\\

\begin{align}
\text{max} & \quad 8 x_1+5 x_2+6 x_3+9 x_4+7 x_5+9 x_6+6 x_7+5 x_8  + 6 x_9 \nonumber \\ 
\text{s.t.} & \quad 5 x_1+ 3 x_2+4 x_3+2 x_4+7 x_5+3 x_6+4 x_7+6 x_8 + x_9 \leq 30 \nonumber \\
&\quad 2 x_1 \leq 16\nonumber\\ 
& \quad  3 x_4 \leq 27 \nonumber\\
& \quad  10 x_7 \leq 60 \nonumber\\
& \quad x_i \geq 0 \quad \forall i \in \{1, \cdots,9\} \quad \text{and} \quad x_i \in \mathbb{Z} \quad \forall i \in \{1, \cdots,9\} \nonumber
\end{align} \nonumber


\end{document}